\documentclass[12pt, a4paper]{report}

%---- GENERAL ---%
\usepackage[utf8x]{inputenc} % Encodage du fichier
\usepackage[T1]{fontenc}  % Encodage des polices du document
\usepackage[english, francais]{babel} % Langages utilisés
\usepackage{lmodern} % Police pour rendre le texte sélectionable

\usepackage[top=2.5cm, bottom=2.5cm, left=2.5cm, right=2.5cm]{geometry} % Marges PDF

%---- TABLE DES MATIERES - PRESENTATION ---%
\usepackage[titles]{tocloft} % Package pour paramtérer la table des matières
\renewcommand{\cftchapleader}{\cftdotfill{\cftdotsep}} % Mets des points pour les chapitre et les sections
\usepackage[colorlinks=true,urlcolor=blue,pdfstartview=FitH,linkcolor=blue,hyperfootnotes=true]{hyperref} % Liens

%---- GLOSSAIRE ---%
\usepackage{glossaries}
\makeglossaries

%---- STRUCTURES DE DONNEES & MISE EN FORME ---%
\usepackage{enumerate} % Listes
\usepackage{graphicx} % Figures
\usepackage{color} % Couleur
\usepackage{float} % Utilisation de l'option [H] pour forcer l'insertion d'une image après le texte
\definecolor{darkblue}{RGB}{0,0,190}
\definecolor{lightgray}{gray}{0.70}
\definecolor{darkgreen}{RGB}{0,150,50}

%---- HEADERS & FOOTERS ---%
\usepackage{fancyhdr}
\pagestyle{fancy}
\renewcommand{\sectionmark}[1]{\markright{#1}}
\fancyhf{}

\fancyhead[RE]{\chaptername{} \thechapter{} - \nouppercase{\rightmark}} % Even page - Right side
\fancyhead[LE]{\thepage} % Even page - Left side
\fancyhead[RO]{\thepage}% Odd page - Rigth side
\fancyhead[LO]{\chaptername{} \thechapter{} - \nouppercase{\rightmark}} % Odd page - Left side
%\fancyfoot[C]{footer} % Footer - Center

\newcommand{\clearemptydoublepage}{\newpage{\pagestyle{empty}\cleardoublepage}} % Supprime les en-têtes et pieds de page des pages vides (type book)

%---- CHAPTERS ---%
\usepackage[Lenny]{fncychap}

%---- CODES SOURCES & ALGORITHMES ---%
\usepackage{listings} % Codes sources

% [style = sourceC] for the C codes
\lstdefinestyle{sourceC}{language=C, basicstyle=\small\color{black}\ttfamily, classoffset=0, keywordstyle=\color{blue}\bfseries, classoffset=1, morekeywords={NULL}, keywordstyle=\color{magenta}\bfseries, commentstyle=\color{brightblue}\itshape, stringstyle=\color{darkgreen}, showstringspaces=false, tabsize=3, framexleftmargin=2mm, frame=shadowbox, rulesepcolor=\color{lightgray}, breaklines=true, emph={main, printf, scanf, FILE, fopen, fscanf, fprintf, fclose, rand}, emphstyle=\color{darkblue}\bfseries,columns=
fullflexible, flexiblecolumns=true, upquote=true, keepspaces=true}

% [style = inlineSourceC] for C expression into text
\lstdefinestyle{inlineSourceC}{language=C, basicstyle=\footnotesize\color{black}\ttfamily, classoffset=0, keywordstyle=\color{blue}\bfseries, classoffset=1, morekeywords={NULL}, keywordstyle=\color{magenta}\bfseries, commentstyle=\color{brightblue}\itshape, stringstyle=\color{darkgreen}, showstringspaces=false, tabsize=3, breaklines=true, emph={main, printf, scanf, FILE, fopen, fscanf, fprintf, fclose, rand}, emphstyle=\color{darkblue}\bfseries,columns=fullflexible, keepspaces=true,upquote=true}

% [style = sourceJava] for the Java codes
\lstdefinestyle{sourceJava}{language=Java, basicstyle=\small\color{black}\ttfamily, classoffset=0, keywordstyle=\color{darkgreen}\bfseries, commentstyle=\color{brightblue}\itshape, stringstyle=\color{darkgreen}, showstringspaces=false, tabsize=3, framexleftmargin=2mm, frame=shadowbox, rulesepcolor=\color{lightgray}, breaklines=true, emph={main, printf, FILE, fopen, fscanf, fprintf, fclose,new,return,this,super,null,break,continue,if,else,switch,case,default,do,while,for,break,new,return}, emphstyle=\color{darkred}\bfseries,columns=fullflexible,upquote=true}

% [style = sourceJava] for the Java codes
\lstdefinestyle{inlineSourceJava}{style=sourceJava, basicstyle=\normalsize\color{black}\ttfamily}

% [style = msgTerminal] for shell codes on black background
\lstdefinestyle{msgTerminal}{language=sh, basicstyle=\small\color{white}\ttfamily, keywordstyle=\color{white}, commentstyle=\color{white}\itshape, stringstyle=\color{white}, showstringspaces=false, framexleftmargin=3mm, xleftmargin=3mm, frame=none, tabsize=3, backgroundcolor=\color{black}, rulecolor=\color{black}, breaklines=true,columns=fullflexible,upquote=true}

% [style = msgTerminalW] for shell codes on white background
\lstdefinestyle{msgTerminalW}{language=sh, basicstyle=\small\color{black}\ttfamily, keywordstyle=\color{black}, commentstyle=\color{black}\itshape, stringstyle=\color{black}, showstringspaces=false, framexleftmargin=3mm, xleftmargin=3mm, frame=none, tabsize=3, backgroundcolor=\color{white}, rulecolor=\color{white}, breaklines=true,columns=fullflexible,upquote=true}

% [style = inlineTerminal] for shell expression into text
\lstdefinestyle{inlineTerminal}{language=sh, basicstyle=\footnotesize\color{black}\ttfamily\bfseries, keywordstyle=\color{black}, commentstyle=\color{black}\itshape, stringstyle=\color{black}\itshape, showstringspaces=false, tabsize=3, breaklines=true,upquote=true}


\usepackage[french]{algorithm2e} % Algorithmes 
\newenvironment{algo}[0]{\vrule~\vrule\itshape \begin{algorithm}[H] }{\normalfont\end{algorithm}}
\newcommand{\var}[1]{\textnormal{\texttt{#1}}}
\RestyleAlgo{plain}
\SetKwSwitch{Selon}{Cas}{Autre}{selon}{faire}{cas où}{autres cas}{fin}{fin} %Si la compilation bloque à algorithm2e, tenter la ligne du dessous à la place de celle-là
%\SetKwSwitch{Selon}{Cas}{Autre}{selon}{faire}{cas où}{autres cas}{fin}
\SetKwRepeat{Repeter}{r\'ep\'eter}{tant que}
\SetKwFor{uTq}{tant que}{faire}{} % Tant que sans fin (équivaut a uSi)
\SetKwFor{oTq}{}{}{fin} % Tant que déjà ouvert, donc sans balise de depart
\SetKwIF{oSi}{}{}{}{}{}{}{fin}
\SetKwIF{oFn}{}{}{fonction}{}{}{}{fin}
\SetKwFunction{KwFn}{Fn}


%---- DOCUMENT ---%
% COMPILATION : 
% pdflatex squelette_rapport.tex
% bibtex squelette_rapport
% makeindex squelette_rapport
% pdflatex squelette_rapport.tex
% pdflatex squelette_rapport.tex

\title{\textbf{Portage architecture Arduino vers ESP32} \\ \Large{--- Épita --- \\ \textit{Équipe Robotique d'Exploration}} \\ \vspace{1,5cm} \includegraphics[width=0.3\textwidth]{./img/logo_epita} \includegraphics[width=0.2\textwidth]{./img/logo_equipe_robotique_exploration} \vspace{1,5cm}}
\date{\today}
\author{Tony BERNIS \and Valentin DUMOUSSET \and Julien LE QUANG \and Merlin VOTAT}

\begin{document}
	\maketitle
	
	

	%-----------------------
	\chapter*{Résumé}
	\addcontentsline{toc}{chapter}{Résumé}
	%-----------------------	

Micro ROS (Robot Operating System) est un framework pour faciliter le développement robotique et qui sert d'interface entre 
les microcontrôleurs.
Il permet ainsi l'abstraction du matériel, la transmission de messages etc.
Micro ROS n'est pas disponible sur toutes les cartes Arduino. Dans notre projet, nous avons choisi une carte Arduino Nano RP2040.
L'objectif de notre projet était dans un premier temps de faire bouger l'arraignée et dans un second temps travailler sur 
la synchronisation avec une autre arraignée.

	%-----------------------
	% TABLE DES MATIERES %
	\tableofcontents
	%-----------------------

	%-----------------------%
	%     INTRODUCTION      %
	%-----------------------%
	
	%-----------------------
	\chapter{Introduction}
	%-----------------------

ROS2 (Robot Operating System) est un système d'exploitation open source pour les robots. 
Il est conçu pour faciliter le développement de logiciels pour les robots en offrant des bibliothèques, 
des outils et des standards communs pour la robotique. ROS2 offre une infrastructure logicielle pour la programmation de robots, 
y compris des fonctionnalités telles que la communication entre les composants logiciels, la gestion des données, 
la planification et l'exécution de tâches, la visualisation de données, et bien plus encore. 
ROS2 est utilisé dans de nombreux types de robots, des petits robots de service aux grands robots industriels. 
\linebreak

Micro-ROS est une version légère de ROS2 conçue pour fonctionner sur des systèmes embarqués avec 
des ressources limitées en termes de calcul et de mémoire. Micro-ROS offre une infrastructure logicielle pour la programmation 
de robots embarqués, y compris des fonctionnalités similaires à celles de ROS2, telles que la communication entre les composants 
logiciels, la gestion des données, la planification et l'exécution de tâches. 
Micro-ROS est conçu pour être facile à utiliser et à intégrer dans des projets de robotique embarqué, en offrant une grande 
flexibilité et une grande efficacité en termes de ressources. 
\linebreak

Le but de notre projet était d’explorer les différentes options qui nous permettraient d’adapter le code des araignées 
mécaniques sur l’architecture Micro-Ros. 
\linebreak

Il y a plusieurs avantages à utiliser Micro-ROS pour programmer une araignée mécanique. 
Tout d'abord, Micro-ROS est conçu pour fonctionner sur des systèmes embarqués avec des ressources limitées en termes 
de calcul et de mémoire, ce qui est idéal pour une araignée mécanique qui a des contraintes en termes de taille et de poids. 
En outre, Micro-ROS offre une infrastructure logicielle pour la programmation de robots. 
Enfin, Micro-ROS est une version légère de ROS2, ce qui signifie qu'il est moins gourmand en ressources que ROS2 tout en offrant 
les mêmes fonctionnalités de base. Cela peut améliorer les performances de l'araignée mécanique en utilisant moins de ressources 
pour exécuter le logiciel. 
\linebreak

Dans ce rapport, nous allons présenter toutes les recherches que nous avons effectué pendant ce projet, les différentes 
solutions que nous avons conceptualisées ainsi que la solution que nous avons choisie et sa mise en œuvre. 
\linebreak


	%-----------------------%
	%    ETAT DE L'ART      %
	%-----------------------%

	%-----------------------
	\chapter{État de l'Art}
	\thispagestyle{empty}
	%-----------------------
	
Introduction de l'état de l'art - Nous utilisons une manette Xbox pour donner les ordres à notre arraignée.
Voici ce que nous avons réussi à faire :

Utilisation d'une référénce~\cite{NOM00}.


Ut enim ad minima veniam, quis nostrum exercitationem ullam corporis suscipit laboriosam, nisi ut aliquid ex ea commodi consequatur? Quis autem vel eum iure reprehenderit qui in ea voluptate velit esse quam nihil molestiae consequatur, vel illum qui dolorem eum fugiat quo voluptas nulla pariatur?

		%-----------------------
		\section{Les mouvements}
		\label{les_mouvements}
		%-----------------------

// Mettre une photo pour chaque mvt		

sit : faire asseoir l'arraignée

stand : met debout l'arraignée

walk forward, walk back, walk left, walk right : bouge l'arraignée dans les 4 directions

turn left, turn right : tourne l'araignée à gauche et à droite

rave (vague)

flex

hello

applaud sim

applaud wave

applaud

spoutnik

cross

pls : position par défaut

bolting

wink

Utilisation d'une autre référénce~\cite{NOM05}. Et encore une autre~\cite{NOM12}. Ou bien plusieurs~\cite{WEB13, NOM09, NOM07}.
Lorem ipsum dolor sit amet, consectetur adipisicing elit, sed do eiusmod tempor incididunt ut labore et dolore magna aliqua. Ut enim ad minim veniam, quis nostrud exercitation ullamco laboris nisi ut aliquip ex ea commodo consequat. Duis aute irure dolor in reprehenderit in voluptate velit esse cillum dolore eu fugiat nulla pariatur. Excepteur sint occaecat cupidatat non proident, sunt in culpa qui officia deserunt mollit anim id est laborum.

Figure~:

\includegraphics[width=0.3\textwidth]{./img/logo_equipe_robotique_exploration}

Sed ut perspiciatis unde omnis iste natus error sit voluptatem accusantium doloremque laudantium, totam rem aperiam, eaque ipsa quae ab illo inventore veritatis et quasi architecto beatae vitae dicta sunt explicabo. Nemo enim ipsam voluptatem quia voluptas sit aspernatur aut odit aut fugit, sed quia consequuntur magni dolores eos qui ratione voluptatem sequi nesciunt. Neque porro quisquam est, qui dolorem ipsum quia dolor sit amet, consectetur, adipisci velit, sed quia non numquam eius modi tempora incidunt ut labore et dolore magnam aliquam quaerat voluptatem. 

\begin{figure}
\begin{center}
	\includegraphics[width=0.3\textwidth]{./img/logo_equipe_robotique_exploration}
	\caption{Légende de l'image}
	\label{fig:image1}
\end{center}
\end{figure}


Ut enim ad minima veniam, quis nostrum exercitationem ullam corporis suscipit laboriosam, nisi ut aliquid ex ea commodi consequatur? Quis autem vel eum iure reprehenderit qui in ea voluptate velit esse quam nihil molestiae consequatur, vel illum qui dolorem eum fugiat quo voluptas nulla pariatur?

At vero eos et accusamus et iusto odio dignissimos ducimus qui blanditiis praesentium voluptatum deleniti atque corrupti quos dolores et quas molestias excepturi sint occaecati cupiditate non provident, similique sunt in culpa qui officia deserunt mollitia animi, id est laborum et dolorum fuga. Et harum quidem rerum facilis est et expedita distinctio. Nam libero tempore, cum soluta nobis est eligendi optio cumque nihil impedit quo minus id quod maxime placeat facere possimus, omnis voluptas assumenda est, omnis dolor repellendus. Temporibus autem quibusdam et aut officiis debitis aut rerum necessitatibus saepe eveniet ut et voluptates repudiandae sint et molestiae non recusandae. Itaque earum rerum hic tenetur a sapiente delectus, ut aut reiciendis voluptatibus maiores alias consequatur aut perferendis doloribus asperiores repellat.
		
		%-----------------------
		\section{Augmentation vitesse mouvements}
		%-----------------------

Pour augmenter ou diminuer la vitesse des mouvements, il faut appuyer sur les joysticks de la manette.

Celui de gauche permet d'augmenter le délai entre les instructions et celui de droite permet de le diminuer.
		
Référence interne, voir la section \ref{les_mouvements} page~\pageref{les_mouvements}. % fait référence à l'endroit où on a posé le label correspondant au nom (\label{les_mouvements})
Lorem ipsum dolor sit amet, consectetur adipisicing elit, sed do eiusmod tempor incididunt ut labore et dolore magna aliqua. Ut enim ad minim veniam, quis nostrud exercitation ullamco laboris nisi ut aliquip ex ea commodo consequat. Duis aute irure dolor in reprehenderit in voluptate velit esse cillum dolore eu fugiat nulla pariatur. Excepteur sint occaecat cupidatat non proident, sunt in culpa qui officia deserunt mollit anim id est laborum.

\begin{lstlisting}[style=sourceC]
#include <stdio.h>

int main(void) {
	printf("Hello World!\n");
	return 0;
}
\end{lstlisting}		

 
Sed ut perspiciatis unde omnis iste natus error sit voluptatem accusantium doloremque laudantium, totam rem aperiam, eaque ipsa quae ab illo inventore veritatis et quasi architecto beatae vitae dicta sunt explicabo. Nemo enim ipsam voluptatem quia voluptas sit aspernatur aut odit aut fugit, sed quia consequuntur magni dolores eos qui ratione voluptatem sequi nesciunt. 

Code \lstinline[style=sourceC]!#include <stdio.h>! en ligne.


Ut enim ad minima veniam, quis nostrum exercitationem ullam corporis suscipit laboriosam, nisi ut aliquid ex ea commodi consequatur? Quis autem vel eum iure reprehenderit qui in ea voluptate velit esse quam nihil molestiae consequatur, vel illum qui dolorem eum fugiat quo voluptas nulla pariatur?

Code en provenance d'un fichier~:

\lstinputlisting[style=sourceC]{src/helloWorld.c} 

At vero eos et accusamus et iusto odio dignissimos ducimus qui blanditiis praesentium voluptatum deleniti atque corrupti quos dolores et quas molestias excepturi sint occaecati cupiditate non provident, similique sunt in culpa qui officia deserunt mollitia animi, id est laborum et dolorum fuga. Et harum quidem rerum facilis est et expedita distinctio. Nam libero tempore, cum soluta nobis est eligendi optio cumque nihil impedit quo minus id quod maxime placeat facere possimus, omnis voluptas assumenda est, omnis dolor repellendus. 

\begin{algorithm}[H]	
	Une instruction \\
	\Tq{La condition est vraie}{
		\eSi{Une autre condition est vraie}{
			Une instruction
		}{
			Une autre instruction
		}
	}
	\Si{Une condition est vraie}{
		Une instruction \\
		Une autre instruction \\	
	}
	\Repeter{La condition est vraie}{
		Une instruction
	}
	
\end{algorithm}


Temporibus autem quibusdam et aut officiis debitis aut rerum necessitatibus saepe eveniet ut et voluptates repudiandae sint et molestiae non recusandae. Itaque earum rerum hic tenetur a sapiente delectus, ut aut reiciendis voluptatibus maiores alias consequatur aut perferendis doloribus asperiores repellat.
			

Considérez par exemple l'équation \ref{eq:equation1} page~\pageref{eq:equation1}~:

		%-----------------------
		\section{Synchronisation arraignées}
		%-----------------------

Les arraignées sont synchros. Elles sont connectés par wifi à l'ordinateur qui envoient les messages en UDP.

\begin{equation}
\left(\frac{x^2}{y^3}\right)
\label{eq:equation1}
\end{equation}

\clearemptydoublepage


			
	%-----------------------%
	% ETUDE DE FAISABILITE  %
	%-----------------------%

	%-----------------------
	\chapter{Implémentation}
	%-----------------------

Introduction de l'étude de faisabilité - Lorem ipsum dolor sit amet, consectetur adipisicing elit, sed do eiusmod tempor incididunt ut labore et dolore magna aliqua. Ut enim ad minim veniam, quis nostrud exercitation ullamco laboris nisi ut aliquip ex ea commodo consequat. Duis aute irure dolor in reprehenderit in voluptate velit esse cillum dolore eu fugiat nulla pariatur. Excepteur sint occaecat cupidatat non proident, sunt in culpa qui officia deserunt mollit anim id est laborum.

	Sed ut perspiciatis unde omnis iste natus error sit voluptatem accusantium doloremque laudantium, totam rem aperiam, eaque ipsa quae ab illo inventore veritatis et quasi architecto beatae vitae dicta sunt explicabo. Nemo enim ipsam voluptatem quia voluptas sit aspernatur aut odit aut fugit, sed quia consequuntur magni dolores eos qui ratione voluptatem sequi nesciunt. Neque porro quisquam est, qui dolorem ipsum quia dolor sit amet, consectetur, adipisci velit, sed quia non numquam eius modi tempora incidunt ut labore et dolore magnam aliquam quaerat voluptatem.
	
		%-----------------------
		\section{Implémentation technique}
		%-----------------------

// A confirmer / compléter par notre expert Merlin

Les ordres sont envoyées dans une queue et il y a un publisher et un récepteur.

		%-----------------------
		\section{Limites et problèmes rencontrés}
		%-----------------------

			%-----------------------
			\subsection{Limites}
			%-----------------------

Alimentation : cartes besoin d'une batterie externe supplémentaire 

Nous avons pensé à faire un support mais nous n'avons pas eu le temps (le support n'étant pas un élément essentiel).

			%-----------------------
			\subsection{Problèmes rencontrés}
			%-----------------------

Servomoteur : on a besoin de la lib RP2040\_ISR\_Servo (\url{https://github.com/khoih-prog/RP2040_ISR_Servo}) car les mouvements envoient une instruction en dehors de la loop principale

Impossible d'utiliser l'Arduino IDE (car il manque un time.h).

Nous avons utilisé VSCode avec PlatformIO qui est une extension de VSCode. Il est compatible Windows, Mac et Linux.

PlatformIO est designé pour faire marcher énormément de cartes et il faut donc spécifier la carte sur laquelle on travaille. Il se charge ensuite de récupèrer les libs.




	%-----------------------%
	%       CONCLUSION      %
	%-----------------------%
	
	%-----------------------
	\chapter{Conclusion}
	%-----------------------

Conclusion du document - Ce projet nous a beaucoup apportés.

Beaucoup de connaissances : la modélisation 3d, des notions en électronique et robotique

Nous avons atteint nos objectifs : mvt des arraignées + synchro

\newglossaryentry{word1}{name=mot en entier, description={Définition du mot}}
\newglossaryentry{word2}{name=anotherword, description={Bla bla}}

Voici une phrase qui utilise un \gls{word1} du glossaire et encore un autre \gls{word2}.

% -------------------------------------------------------------------------------------------------
% l'intégration du glossaire et de la bibliographie est K.O dans le squelette qui nous a été fourni
% nous avons donc inséré un chapitre "customisé"

	%-----------------------%
	%      GLOSSAIRE        %
	%-----------------------%
	%-----------------------
	\chapter*{\centerline{\MakeUppercase{Glossaire}}}
	%-----------------------


\begin{itemize}
	\item[$\bullet$] Ros2 : ROS2 (Robot Operating System 2) est un système d'exploitation open source pour les robots, 
    basé sur une architecture modulaire et un modèle de communication basé sur les services et les requêtes, qui permet de 
    développer et d'exécuter des applications de robotique de manière flexible et scalable. 

    \item[$\bullet$] Micro-ROS : Micro-ROS est une implémentation légère de ROS2 (Robot Operating System 2) conçue pour être 
    exécutée sur des périphériques embarqués tels que les microcontrôleurs et les cartes de développement. 
 
    \item[$\bullet$] Esp32 : ESP32 est un microcontrôleur de la famille ESP de Espressif Systems, conçu pour être utilisé 
    dans des applications embarquées telles que les objets connectés et les capteurs. 
     
    \item[$\bullet$] Arduino Nano : Arduino Nano est une carte de développement microcontrôleur basée sur un microcontrôleur Atmel AVR. 
    Elle est conçue pour être petite et portable, avec un format de circuit imprimé compact qui peut être facilement intégré dans des 
    projets de robotique et d'objets connectés. 
     
    \item[$\bullet$] OpenSCAD : OpenSCAD est un logiciel de modélisation 3D de type "constructif", qui permet de créer des modèles 3D 
    en utilisant une syntaxe de script plutôt que par une interface graphique. 
\end{itemize}

%\deftranslation{Glossary}{Glossaire}
%\addcontentsline{toc}{chapter}{Glossaire} % Rajoute la partie dans le sommaire
%\printglossaries
	
	%-----------------------%
	%      BIBLIOGRAPHIE    %
	%-----------------------%

	%-----------------------
	\chapter*{\centerline{\MakeUppercase{Bibliographie}}}
	%-----------------------


\url{https://learn.sunfounder.com/category/quadruped-crawling-robot-v2-0-for-arduino/}
\linebreak

\url{https://micro.ros.org/}
\linebreak

\url{https://github.com/micro-ROS/micro_ros_arduino}
\linebreak

\url{https://github.com/micro-ROS/micro_ros_espidf_component}
\linebreak

\url{https://docs.arduino.cc/hardware/nano-rp2040-connect}
\linebreak

\url{https://store.arduino.cc/products/portenta-h7}
\linebreak

\url{https://micro.ros.org/blog/2020/08/27/esp32/}
\linebreak

\url{https://husarnet.com/blog/esp32-microros}
\linebreak

\url{https://www.hadabot.com/setup-esp32-to-work-with-ros2.html}
\linebreak

\url{https://docs.ros.org/en/humble/Installation/Ubuntu-Install-Debians.html}
\linebreak

\url{https://gist.github.com/Redstone-RM/}
\linebreak

\url{https://github.com/micro-ROS/micro_ros_arduino/blob/galactic/examples/micro-ros_publisher_wifi/micro-ros_publisher_wifi.ino}
\linebreak
%\addcontentsline{toc}{chapter}{Bibliographie} % Rajoute la partie dans le sommaire
%\bibliographystyle{alpha}
%\bibliography{biblio/bibliographie.bib}

	
\end{document}