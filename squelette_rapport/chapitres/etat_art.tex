	%-----------------------
	\chapter{État de l'Art}
	\thispagestyle{empty}
	%-----------------------
	
Introduction de l'état de l'art - Nous utilisons une manette Xbox pour donner les ordres à notre arraignée.
Voici ce que nous avons réussi à faire :

Utilisation d'une référénce~\cite{NOM00}.

		%-----------------------
		\section{Les solutions}
		%-----------------------

			%-----------------------
			\subsection{Solution 1}
			%-----------------------

A l’heure actuelle, Micro-ros n’est supporté que par 2 modèles de cartes Arduino, Arduino Portenta H7 M7 Core
\url{https://store.arduino.cc/portenta-h7} et Arduino Nano RP2040 Connect \url{https://docs.arduino.cc/hardware/nano-rp2040-connect}. 
Il est probablement possible d’adapter Micro-ROS à la carte Arduino que nous avons à notre disposition mais cela risque de nous 
prendre plus de temps que nous n’avons à disposition.  
\linebreak
	
L'Arduino Nano RP2040 Connect est une carte de développement basée sur le processeur ARM Cortex-M0+ RP2040 conçue par Raspberry Pi. 
Elle est conçue pour être petite, flexible et facile à utiliser pour les projets de robotique et d'Internet des objets (IoT). 
L'Arduino Nano RP2040 Connect est équipée de 20 broches de E/S numériques, de 6 broches PWM, de 6 broches analogiques, 
d'un port USB-C pour l'alimentation et la communication, d'un connecteur de batterie, d'un connecteur de microphone, 
d'un connecteur de haut-parleur et d'un connecteur de bouton-poussoir. La carte est également compatible avec les bibliothèques Arduino 
pour faciliter le développement de logiciels pour les projets de robotique et d'IoT. 
Après nos différentes recherches, nous avons conclu que ce serait la carte idéale pour mettre en place le projet. 
\linebreak 

IMAGES DU WORD A REMETTRE

			%-----------------------
			\subsection{Solution 2}
			%-----------------------

Nous avons cependant pensé à une solution alternative. Nous pouvons utiliser une carte ESP32 qui est déjà compatible 
avec l’architecture Micro-ROS ( \url{https://micro.ros.org/blog/2020/08/27/esp32} ).  
\linebreak

ESP32 est une puce de microcontrôleur à double cœur conçue par Espressif Systems. Elle est principalement utilisée 
dans les applications de robotique, d'Internet des objets (IoT) et de réseaux locaux sans fil (Wi-Fi et Bluetooth). 
La puce ESP32 est équipée d'un processeur principal Xtensa Dual-Core LX6, d'un processeur coprocesseur ultra-basse consommation, 
d'une mémoire SRAM de 520 Ko, d'une mémoire flash de 16 Mo, d'un module Wi-Fi 802.11b/g/n/e/i et d'un module Bluetooth v4.2. 
La puce ESP32 offre également de nombreux E/S numériques, analogiques et PWM, ainsi que des fonctionnalités avancées telles 
que le traitement du signal numérique, le traitement en temps réel, l'interface de caméra et l'interface de bus série. 
En raison de ses performances et de ses fonctionnalités avancées, la puce ESP32 est largement utilisée dans de nombreux projets de 
robotique et d'IoT.
\linebreak
	
On peut installer cette carte sur notre robot et la brancher directement en SPI (Port Radio ?) avec la carte Arduino nano pour 
ne pas avoir à la connecter en radio ou en wifi.  
L’ESP32 s’occupera de faire tous les contrôles et enverra des ordres a la carte Arduino qui se contentera de les exécuter 
Les problèmes que l’on pourrait rencontrer avec cette méthode sont le manque de place sur le robot, et un poids peut être trop importants. 
Ainsi qu’un manque d’alimentions, l’alimentations actuelle du robot ne sera peut-être pas suffisante pour alimenter les deux cartes. 
\linebreak
	
(Parler des exo ROS2) 

		%-----------------------
		\section{Les mouvements}
		\label{les_mouvements}
		%-----------------------

// Mettre une photo pour chaque mvt		

sit : faire asseoir l'arraignée

stand : met debout l'arraignée

walk forward, walk back, walk left, walk right : bouge l'arraignée dans les 4 directions

turn left, turn right : tourne l'araignée à gauche et à droite

rave (vague)

flex

hello

applaud sim

applaud wave

applaud

spoutnik

cross

pls : position par défaut

bolting

wink

Utilisation d'une autre référénce~\cite{NOM05}. Et encore une autre~\cite{NOM12}. Ou bien plusieurs~\cite{WEB13, NOM09, NOM07}.
Lorem ipsum dolor sit amet, consectetur adipisicing elit, sed do eiusmod tempor incididunt ut labore et dolore magna aliqua. Ut enim ad minim veniam, quis nostrud exercitation ullamco laboris nisi ut aliquip ex ea commodo consequat. Duis aute irure dolor in reprehenderit in voluptate velit esse cillum dolore eu fugiat nulla pariatur. Excepteur sint occaecat cupidatat non proident, sunt in culpa qui officia deserunt mollit anim id est laborum.

Figure~:

\includegraphics[width=0.3\textwidth]{./img/logo_equipe_robotique_exploration}

Sed ut perspiciatis unde omnis iste natus error sit voluptatem accusantium doloremque laudantium, totam rem aperiam, eaque ipsa quae ab illo inventore veritatis et quasi architecto beatae vitae dicta sunt explicabo. Nemo enim ipsam voluptatem quia voluptas sit aspernatur aut odit aut fugit, sed quia consequuntur magni dolores eos qui ratione voluptatem sequi nesciunt. Neque porro quisquam est, qui dolorem ipsum quia dolor sit amet, consectetur, adipisci velit, sed quia non numquam eius modi tempora incidunt ut labore et dolore magnam aliquam quaerat voluptatem. 

\begin{figure}
\begin{center}
	\includegraphics[width=0.3\textwidth]{./img/logo_equipe_robotique_exploration}
	\caption{Légende de l'image}
	\label{fig:image1}
\end{center}
\end{figure}


Ut enim ad minima veniam, quis nostrum exercitationem ullam corporis suscipit laboriosam, nisi ut aliquid ex ea commodi consequatur? Quis autem vel eum iure reprehenderit qui in ea voluptate velit esse quam nihil molestiae consequatur, vel illum qui dolorem eum fugiat quo voluptas nulla pariatur?

At vero eos et accusamus et iusto odio dignissimos ducimus qui blanditiis praesentium voluptatum deleniti atque corrupti quos dolores et quas molestias excepturi sint occaecati cupiditate non provident, similique sunt in culpa qui officia deserunt mollitia animi, id est laborum et dolorum fuga. Et harum quidem rerum facilis est et expedita distinctio. Nam libero tempore, cum soluta nobis est eligendi optio cumque nihil impedit quo minus id quod maxime placeat facere possimus, omnis voluptas assumenda est, omnis dolor repellendus. Temporibus autem quibusdam et aut officiis debitis aut rerum necessitatibus saepe eveniet ut et voluptates repudiandae sint et molestiae non recusandae. Itaque earum rerum hic tenetur a sapiente delectus, ut aut reiciendis voluptatibus maiores alias consequatur aut perferendis doloribus asperiores repellat.
		
		%-----------------------
		\section{Augmentation vitesse mouvements}
		%-----------------------

Pour augmenter ou diminuer la vitesse des mouvements, il faut appuyer sur les joysticks de la manette.

Celui de gauche permet d'augmenter le délai entre les instructions et celui de droite permet de le diminuer.
		
Référence interne, voir la section \ref{les_mouvements} page~\pageref{les_mouvements}. % fait référence à l'endroit où on a posé le label correspondant au nom (\label{les_mouvements})
Lorem ipsum dolor sit amet, consectetur adipisicing elit, sed do eiusmod tempor incididunt ut labore et dolore magna aliqua. Ut enim ad minim veniam, quis nostrud exercitation ullamco laboris nisi ut aliquip ex ea commodo consequat. Duis aute irure dolor in reprehenderit in voluptate velit esse cillum dolore eu fugiat nulla pariatur. Excepteur sint occaecat cupidatat non proident, sunt in culpa qui officia deserunt mollit anim id est laborum.

\begin{lstlisting}[style=sourceC]
#include <stdio.h>

int main(void) {
	printf("Hello World!\n");
	return 0;
}
\end{lstlisting}		

 
Sed ut perspiciatis unde omnis iste natus error sit voluptatem accusantium doloremque laudantium, totam rem aperiam, eaque ipsa quae ab illo inventore veritatis et quasi architecto beatae vitae dicta sunt explicabo. Nemo enim ipsam voluptatem quia voluptas sit aspernatur aut odit aut fugit, sed quia consequuntur magni dolores eos qui ratione voluptatem sequi nesciunt. 

Code \lstinline[style=sourceC]!#include <stdio.h>! en ligne.


Ut enim ad minima veniam, quis nostrum exercitationem ullam corporis suscipit laboriosam, nisi ut aliquid ex ea commodi consequatur? Quis autem vel eum iure reprehenderit qui in ea voluptate velit esse quam nihil molestiae consequatur, vel illum qui dolorem eum fugiat quo voluptas nulla pariatur?

Code en provenance d'un fichier~:

\lstinputlisting[style=sourceC]{src/helloWorld.c} 

At vero eos et accusamus et iusto odio dignissimos ducimus qui blanditiis praesentium voluptatum deleniti atque corrupti quos dolores et quas molestias excepturi sint occaecati cupiditate non provident, similique sunt in culpa qui officia deserunt mollitia animi, id est laborum et dolorum fuga. Et harum quidem rerum facilis est et expedita distinctio. Nam libero tempore, cum soluta nobis est eligendi optio cumque nihil impedit quo minus id quod maxime placeat facere possimus, omnis voluptas assumenda est, omnis dolor repellendus. 

\begin{algorithm}[H]	
	Une instruction \\
	\Tq{La condition est vraie}{
		\eSi{Une autre condition est vraie}{
			Une instruction
		}{
			Une autre instruction
		}
	}
	\Si{Une condition est vraie}{
		Une instruction \\
		Une autre instruction \\	
	}
	\Repeter{La condition est vraie}{
		Une instruction
	}
	
\end{algorithm}


Temporibus autem quibusdam et aut officiis debitis aut rerum necessitatibus saepe eveniet ut et voluptates repudiandae sint et molestiae non recusandae. Itaque earum rerum hic tenetur a sapiente delectus, ut aut reiciendis voluptatibus maiores alias consequatur aut perferendis doloribus asperiores repellat.
			

Considérez par exemple l'équation \ref{eq:equation1} page~\pageref{eq:equation1}~:

		%-----------------------
		\section{Synchronisation arraignées}
		%-----------------------

Les arraignées sont synchros. Elles sont connectés par wifi à l'ordinateur qui envoient les messages en UDP.

\begin{equation}
\left(\frac{x^2}{y^3}\right)
\label{eq:equation1}
\end{equation}

\clearemptydoublepage

