	%-----------------------
	\chapter{Introduction}
	%-----------------------

ROS2 (Robot Operating System) est un système d'exploitation open source pour les robots. 
Il est conçu pour faciliter le développement de logiciels pour les robots en offrant des bibliothèques, 
des outils et des standards communs pour la robotique. ROS2 offre une infrastructure logicielle pour la programmation de robots, 
y compris des fonctionnalités telles que la communication entre les composants logiciels, la gestion des données, 
la planification et l'exécution de tâches, la visualisation de données, et bien plus encore. 
ROS2 est utilisé dans de nombreux types de robots, des petits robots de service aux grands robots industriels. 
\linebreak

Micro-ROS est une version légère de ROS2 conçue pour fonctionner sur des systèmes embarqués avec 
des ressources limitées en termes de calcul et de mémoire. Micro-ROS offre une infrastructure logicielle pour la programmation 
de robots embarqués, y compris des fonctionnalités similaires à celles de ROS2, telles que la communication entre les composants 
logiciels, la gestion des données, la planification et l'exécution de tâches. 
Micro-ROS est conçu pour être facile à utiliser et à intégrer dans des projets de robotique embarqué, en offrant une grande 
flexibilité et une grande efficacité en termes de ressources. 
\linebreak

Le but de notre projet était d’explorer les différentes options qui nous permettraient d’adapter le code des araignées 
mécaniques sur l’architecture Micro-Ros. 
\linebreak

Il y a plusieurs avantages à utiliser Micro-ROS pour programmer une araignée mécanique. 
Tout d'abord, Micro-ROS est conçu pour fonctionner sur des systèmes embarqués avec des ressources limitées en termes 
de calcul et de mémoire, ce qui est idéal pour une araignée mécanique qui a des contraintes en termes de taille et de poids. 
En outre, Micro-ROS offre une infrastructure logicielle pour la programmation de robots. 
Enfin, Micro-ROS est une version légère de ROS2, ce qui signifie qu'il est moins gourmand en ressources que ROS2 tout en offrant 
les mêmes fonctionnalités de base. Cela peut améliorer les performances de l'araignée mécanique en utilisant moins de ressources 
pour exécuter le logiciel. 
\linebreak

Dans ce rapport, nous allons présenter toutes les recherches que nous avons effectué pendant ce projet, les différentes 
solutions que nous avons conceptualisées ainsi que la solution que nous avons choisie et sa mise en œuvre. 
\linebreak
